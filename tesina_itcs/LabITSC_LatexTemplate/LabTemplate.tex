\documentclass[9pt,shortpaper,twoside,web]{ieeecolor}
\usepackage{generic}
\usepackage{cite}
\usepackage{amsmath,amssymb,amsfonts}
\usepackage{algorithmic}
\usepackage{graphicx}
\usepackage{textcomp}
\usepackage[font=small]{caption}
\usepackage{booktabs}
\usepackage{multirow}




\markboth{Laboratorio di ingegneria e tecnologia dei sistemi di controllo. Docente: Francesco Smarra.}
{Author \MakeLowercase{\textit{et al.}}: report di laboratorio di ingegneria e tecnologia dei sistemi di controllo}
\begin{document}
\title{Report x: ex. name}
\author{First A. Author, and Second B. Author
\thanks{Laboratorio ITSC, report x, insert here the report realization date.}
\thanks{The next few paragraphs should contain 
the authors' current affiliations and e-mail. For 
example, F. A. Author is with the Universit\`{a} degli Studi dell'Aquila, L'Aquila, Italy (e-mail: author@boulder.nist.gov). }
\thanks{S. B. Author is with 
the Electrical Engineering Department, University of Colorado, Boulder, CO 
80309 USA, on leave from the National Research Institute for Metals, 
Tsukuba, Japan (e-mail: author@nrim.go.jp).}
}

\maketitle

\begin{abstract}
Insert text here.
\end{abstract}

\begin{IEEEkeywords}
Enter key words or phrases in alphabetical order, separated by commas. 
\end{IEEEkeywords}

\section{Descrizione dell'esercitazione}\label{sec:problemDescription}
Insert text here.

\section{Descrizione setup}
Insert text here.

\subsection{Setup 1}
If you have different setups.

\section{Discussione dei risultati}
Insert text here. You can include the code and discuss what you got.

\section{Conclusioni}
Insert text here.



\appendix
Example of figure and table use. In this way you include a figure, as Figure \ref{fig:fig1}.

\begin{figure}[h!]
	\centerline{\includegraphics[width=0.5\columnwidth]{figures/LOGO-generic-web.eps}}
	\caption{UnivAQ logo.}
	\label{fig:fig1}
\end{figure}

In this way you include a table, as Table \ref{tab:houseProperties}.

\begin{table}[h!]
	\centering
	\begin{tabular}{ccccc}
		\toprule
		\multirow{3}{1.4cm}{\centering Structural member} & \multirow{3}{3.7cm}{\centering Layer description (from inside to outside)}       & Thermal      		  \\
														  &                                                                                  & resistance   		  \\
														  &                                             									 & $[(m^2K)/W]$			  \\
		
		
		\midrule
		\multirow{3}*{Wall}                               & Wood-cement                                 									 & $0.308$				  \\ 
														  & Concrete                                    									 & $0.096$    			  \\ 
														  & EPS and graphite                            									 & $6.774$    			  \\
														  &							                    									 &						  \\
		
		
		\multirow{3}*{Floor}        					  & Wood-cement                                 									 & $0.308$     			  \\ 
														  & Polystyrene                                 									 & $6.000$     			  \\ 
														  & Screed                                      									 & $0.027$     			  \\
														  &									    	    									 &						  \\
		
		
		\multirow{5}{1.4cm}{\centering Pitched roof} 	  & Wood-cement                                 									 & $0.308$                \\ 
														  & Polystyrene                                 									 & $6.000$                \\
														  & Screed                                      									 & $0.027$                \\ 
														  & \multirow{2}{3.7cm}{\centering Polyurethane resins and polyisocianurate foams}   & \multirow{2}*{$4.000$} \\
														  &                                             									 &                        \\
		\bottomrule
	\end{tabular}
	\caption{Wood-cement blocks properties.}
	\captionsetup{justification=centering}
	\label{tab:houseProperties}
\end{table}


\end{document}
